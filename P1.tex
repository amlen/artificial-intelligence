\documentclass[a4paper, 12pt]{article}

\usepackage[french]{babel} 
\usepackage[utf8]{inputenc}
\usepackage[T1]{fontenc} 
\usepackage{amsmath}
\usepackage{amssymb}
\usepackage{listings}  
\usepackage{graphicx}
\usepackage[margin=2.5cm]{geometry}
\usepackage{amsmath,amsfonts,amssymb}
\lstset{
language=Java,
breaklines=true
}


\author{Florian \bsc{Thuin}  \and Ivan \bsc{Ahad}}

\title{INGI2261 - Artificial Intelligence - Project 1 : The Numberlink Problem}

\date{\today}

\begin{document}

\maketitle
\section{Introduction}
a
\section{First set of questions}
\subsection{Question 1.}
\textit{In order to perform a search, what are the classes that you must define or extend? Explain precisely why and where they are used inside a treesearch.}
\subsection{Question 2.}
\textit{In the expand method of the class Node, what is the advantage of using a yield instead of building a list and returning it afterwards.}

\subsection{Question 3.}
\textit{Both breadth$\_$first$\_$graph$\_$search and depth$\_$first$\_$graph$\_$search are making a call to the same function. How is their fundamental difference implemented?}
\\
They both call the function"Tree$\_$search". The fundamental difference is that breadth$\_$first$\_$graph$\_$search uses a queue, hence using the First-in First-out principle, while depth$\_$first$\_$graph$\_$search uses a stack, using the Last-in First-out principle. 

\subsection{Question 4.}
\textit{What is difference between the implementation of the graph$\_$search and three$\_$search methods and how does it impact the search methods?}
\\
The graph$\_$search method uses a "closed" entity. It allows the graph$\_$search to know if it has already entered a certain state. When it does, it puts the state in the "closed" list. When 



\section{Conclusion}
a

\end{document}