\bigskip
\textit{This paper is the report of the \textbf{Group 43} for the first assignment
of the course.}

\section{Search Algorithms and their relations}

\begin{enumerate}
 \item Give a consistent heuristic for this problem. Prove that it is admissible.
    \begin{framed}
      \todo[inline]{Réponse à faire}
    \end{framed}
 \item Show on the left maze the states (board positions) that are visited during an
execution of a uniform-cost graph search. We assume that when different states
in the fringe have the smallest value, the algorithm chooses the state with
the smallest coordinate (\textit{i, j}) ((0, 0) being the bottom left position, \textit{i} being the
horizontal index and \textit{j} the vertical one) using a lexicographical order.
    \begin{framed}
      \missingfigure{A maze}
    \end{framed}
  \item Show on the right maze the board positions visited by A graph search with
a manhattan distance heuristic (ignoring walls). A state is visited when it is
selected in the fringe and expanded. When several states have the smallest
path cost, this uniform-cost search visits them in the same lexicographical order
as the one used for uniform-cost graph search.
    \begin{framed}
      \missingfigure{A maze}
    \end{framed}

\end{enumerate}

\section{Sokoban planning problem}

\begin{enumerate}
 \item As illustrated on Figure 3 some situations cannot lead to a solution. Are there
other similar situations? If yes, describe them.
  \begin{framed}
    \todo[inline]{Réponse à faire}
  \end{framed}
  \item Why is it important to identify dead states in your successor function? How
are you going to implement it?
  \item Describe possible (non trivial) heuristic(s) to reach a goal state (with reference
if any). Is(are) your heuristic(s) admissible and/or consistent?
    \begin{framed}
      \todo[inline]{Réponse à faire}
    \end{framed}
  \item Implement this problem. Extend the \textit{Problem} class and implement the necessary
methods and other class(es) if necessary. Your file must be named \textit{sokoban.py}.
You program must print to the standard output a solution to the sokoban instance
given in argument satisfying the described format. You will receive the name
of the instance and you have to read from the two files .init and .goal, for
instance, given instance instance1 you will read from files instance1.init
and instance1.goal .
    \begin{framed}
      \todo[inline]{Réponse à faire}
    \end{framed}
  \item Experiment, compare and analyze informed (\textit{astar\_graph\_search}) and unin-
formed (\textit{breadth\_first\_graph\_search}) graph search of aima-python3 on the 15
instances of sokoban provided. Report in a table the time, the number of ex-
plored nodes and the number of steps to reach the solution. Are the number of
explored nodes always smaller with \textit{astar\_graph\_search}, why?
When no solution can be found by a strategy in a reasonable time (say 5 min),
explain the reason (time-out and/or swap of the memory).
    \begin{framed}
      \todo[inline]{Réponse à faire}
    \end{framed}
  \item What are the performances of your program when you don’t perform dead state
detection?
    \begin{framed}
      \todo[inline]{Réponse à faire}
    \end{framed}
\end{enumerate}
