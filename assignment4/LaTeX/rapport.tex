% Local search
\section{The Wedding Problem}
\begin{enumerate}
    \item Compare the 2 strategies implemented in the previous question between each other and with randomwalk, defined in search.py on the given wedding instances. Report in a table, the computation time, the value of the best solution and the number of steps when the best result was reached. For the second and third strategies, each instance should be tested 10 times to eliminate effects of the randomness on the result. When multiple run of the same instance are executed, report the means of the quantities.
    \begin{framed}
    	\begin{tabular}{l|l|l|l}
    					& maxvalue 		& randomized maxvalue	& randomwalk\\
            \hline
    		64\_16	 	& 5.61 sec		& 6.27 sec		& 6.22 sec\\
    					& 507			& 504.6			& 252.1\\
    					& 32 steps		& 78.6 steps	& 0.2 steps\\
    		99\_33	 	& 21.15 sec		& 21.18 sec		& 19.43 sec\\
    					& 596	 		& 634.8 		& 288.6\\
    					& 38 steps		& 95 steps		& 0.2 steps\\
    		100\_10	 	& 24.03 sec		& 27.22 sec 	& 25.60 sec\\
    					& 1338			& 1379.8		& 410.1\\
    					& 44 steps		& 90 steps		& 2.1 steps\\
    		100\_20	 	& 17.03 sec		& 19.51 sec 	& 17.6 sec\\
    					& 950 			& 954 			& 374.1\\
    					& 52 steps		& 98.7 steps 	& 0.2 steps\\
    	\end{tabular}
    \end{framed}

    \item Answer the following questions:
    	\begin{enumerate}
    		\item What is the best strategy ?
    			\begin{framed}
    				According to the result obtained, the best strategy is randomized maxvalue.
    			\end{framed}
    		\item Why do you think the best strategy beats the other ones ?
    			\begin{framed}
    				Because it offers a good trade-off between diversification and intensification. It always try to proceed toward a better solution but not always through the same way and as such tend to fall less inside local optimum than maxvalue. It is also obviously better than randomwalk as randomwalk doesn't even try to reach a better solution, although thanks to that it will never fall in a local optimum.
    			\end{framed}
    		\item What are the limitations of each strategy in terms of diversification and intensification ?
    			\begin{framed}
    				\textbf{maxvalue} is the best in terms of intensification but worst in terms of diversification, as such it tend to find a good solution rather fast but fail at improving it after a while as it has only a single chance to exit a local optimum. \newline \newline
    				\textbf{randomwalk} is focused on diversification while forgoing any trace of intensification, this result in never falling in local optimum at the cost of never really working toward a better solution. If a better solution is found, it was by luck only.\newline \newline
    				\textbf{randomized maxvalue} offer a good trade of between the two, it always tries to work toward a better solution unlike randomwalk but also has opportunities to avoid falling into local optimum thanks to it's slight diversification. However some local optimum will still be able to trap it as it never consider that two 'bad' swap of guest could lead to a better solution. It is thus also limited although less than the other.
    			\end{framed}
    		\item What is the behaviour of the different techniques when they fall in a local optimum.
    			\begin{framed}
    				\textbf{maxvalue} will try the move leading to the best solution in lexicographic order although it remains degrading comparing to the current best reached. If that solution can lead to a better solution than the best solution found yet will escape the local minimum but it will fall back in otherwise and never be able to leave.\newline \newline
    				\textbf{randomwalk} never really 'falls' in a local optimum as it never really tries to find them in the first place. If it stumble upon one it will just casually leave it unless it is really unlucky and keep going back in (which is unlikely but remains possible)\newline \newline
    				\textbf{randomized maxvalue} will have a choice of five moves to try to exit the local optimum, if one of the moves can lead to a better solution it will thus escape and otherwise it will forever fall back and try to find a way to exit.
    			\end{framed}
    	\end{enumerate}
\end{enumerate}
% Propositionnal logic
\section{Propositional logic}
\begin{enumerate}
    \item For each instance below, give it's number of valid interpretations, i.e. the number of times the sentence is true (considering for each sentence all the proposition variables A, B , C an D).
    	\begin{enumerate}
    		\item $(A \wedge \lnot B) \vee (B \wedge \lnot C)$
    			\begin{framed}
    				\begin{tabular}{c|c|c|c|c|c|c}
    					A	&B	&C	&D	&$(A \wedge \lnot B)$	&$(B \wedge \lnot C)$ &Result\\
    					T	&T	&T	&T	&F			&F			&F\\
    					T	&T	&T	&F	&F			&F			&F\\
    					T	&T	&F	&T	&F			&\textbf{T}	&\textbf{T}\\
    					T	&F	&T	&T	&\textbf{T}	&F			&\textbf{T}\\
    					F	&T	&T	&T	&F			&F			&F\\
    					T	&T	&F	&F	&F			&\textbf{T}	&\textbf{T}\\
    					T	&F	&T	&F	&\textbf{T}	&F			&\textbf{T}\\
    					F	&T	&T	&F	&F			&F			&F\\
    					T	&F	&F	&T	&\textbf{T}	&F			&\textbf{T}\\
    					F	&T	&F	&T	&F			&\textbf{T}	&\textbf{T}\\
    					F	&F	&T	&T	&F			&F			&F\\
    					T	&F	&F	&F	&\textbf{T}	&F			&\textbf{T}\\
    					F	&T	&F	&F	&F			&\textbf{T}	&\textbf{T}\\
    					F	&F	&T	&F	&F			&F			&F\\
    					F	&F	&F	&T	&F			&F			&F\\
    					F	&F	&F	&F	&F			&F			&F\\
    				\end{tabular}
    				\FloatBarrier
    				\vspace{5mm}
    				Thus eight valid interpretations
    			\end{framed}
    		\item $(A \Rightarrow \lnot B) \wedge \lnot(C \vee \lnot D)$
    			\begin{framed}
    				\begin{tabular}{c|c|c|c|c|c|c}
    					A	&B	&C	&D	&$(A \Rightarrow \lnot B)$	&$\lnot(C \vee \lnot D)$ &Result\\
    					T	&T	&T	&T	&F			&F			&F\\
    					T	&T	&T	&F	&F			&F			&F\\
    					T	&T	&F	&T	&F			&\textbf{T}	&F\\
    					T	&F	&T	&T	&\textbf{T}	&F			&F\\
    					F	&T	&T	&T	&\textbf{T}	&F			&F\\
    					T	&T	&F	&F	&F			&F			&F\\
    					T	&F	&T	&F	&\textbf{T}	&F			&F\\
    					F	&T	&T	&F	&\textbf{T}	&F			&F\\
    					T	&F	&F	&T	&\textbf{T}	&\textbf{T}	&\textbf{T}\\
    					F	&T	&F	&T	&\textbf{T}	&\textbf{T}	&\textbf{T}\\
    					F	&F	&T	&T	&\textbf{T}	&F			&F\\
    					T	&F	&F	&F	&\textbf{T}	&F			&F\\
    					F	&T	&F	&F	&\textbf{T}	&F			&F\\
    					F	&F	&T	&F	&\textbf{T}	&F			&F\\
    					F	&F	&F	&T	&\textbf{T}	&\textbf{T}	&\textbf{T}\\
    					F	&F	&F	&F	&\textbf{T}	&F			&F\\
    				\end{tabular}
    				\FloatBarrier
    				\vspace{5mm}
    				Thus three valid interpretations
    			\end{framed}
    	\end{enumerate}
    	
    \item Explain how you can express this problem with propositional logic. What are the variables and how do you translate the relations and the query ? 
    	\begin{framed}
    	\end{framed}
    
    \item Translate your model into Conjonctive Normal Form (CNF) and write it in your report.
    	\begin{framed}
    	\end{framed}
    
\end{enumerate}
